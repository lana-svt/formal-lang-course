\section{Практика 1}

Детали о том, как будет проходить практика.

\subsection{Григорьев С.В.}

Немного про описания языков.
Пописать языковые уравнения, граммтики.
Посмотреть на операции над языками.

Постановка задачи на весь семестр.

Запросы к графовым базам данных.
Контекст задачи, примеры графовых БД (RedisGraph, Neo4j, ...), задача о путях впринципе.

Ссылка на второй конспект.

Задача: реализовать свою "графовую миниБД".

Реализация: оформление, инструменты, языки.

\begin{itemize}
  \item Ограничений на язык реализации нет.
  \item Ограничений на использование библиотек нет. Главное --- не нарушать лицензии и чтобы можно было вносить изменения в библиотеку (при необходимости).
  \item Каждый создёт под решение репозиторий на GitHub и снабжает его всем необходимым: readme, лицензия, CI-сборка с тестированием, инструкции по локальному развёртыванию.
  \item Разработка ведётся в отдельной ветке и когда очередная часть задачи готова к сдаче --- делаем pull request в master и добавляем меня (gsvgit) в ревьюверы.
\end{itemize}


Задачи на дом.
\begin{enumerate}
  \item Выбрать язык программирования, на котором будет вестись разработка.
  \item Создать репозиторий на GitHub.
  \item Настроить CI-сборку и тестирование.
  \item Реализовать подгрузку графов из RDF используя готовые библиотеки.
\end{enumerate}
