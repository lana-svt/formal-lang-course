\section{Лекция 2: Регулярные языки}

Иерархия Хомского. Проблемы с ней. Классы языков.

Грамматики. Системы переписывания.

Регулярные множества. Регулярные языки. Регулярные выражения.

$$
V^* = \bigcup\limits^{i=0}_{\infty} V^i
$$

Конечные автоматы. Система переходов.

Язык, задаваемый автоматом.

Понятие выводимости ($\vdash^*$).

Конфигурация: $\langle \text{Состояние}, \text{Остаток} \rangle$.

Полный автомат и вершина-сток.

Детерминизация, алгоритм Томпсона.

НКА: $\langle \Sigma, Q, s \in Q, T \in Q, \delta: Q \times \Sigma \to 2^Q \rangle$

ДКА: $\langle \Sigma, Q_d, s_d \in Q_d,T_d \in Q_d, \delta_d:Q_d \times \Sigma \to Q_d\rangle$, где:

\begin{itemize}
\item $Q_d=\{q_d \mid q_d \in 2^Q\}$,
\item $s_d=\{s\}$,
\item $T_d=\{q \in Q_d \mid \exists p \in T : p \in q\}$,
\item $\delta_d(q,c)=\{\delta(a,c) \mid a \in q\}$.
\end{itemize}


$\varepsilon$-замыкание.
\begin{enumerate}
\item Транзитивное замыкание отношения $\varepsilon$-перехода.
\item Обработка финальных состояний
\item Добавление переходов: если $\delta(v_0,\varepsilon) = v_1, \delta(v_1,c) = v_2$, то добавим $\delta(v_0,c) = v_2$.
\item Удалим $\varepsilon$-переходы.
\end{enumerate}

Эквивалентность автоматов. Эквивалентность состояний: состояния эквивалентны ели нет различающей строки.

Минимизация.


Теорема Клини об эквивалентности автоматов и регулярных языков.

Построение автомата по регулярному выражению.

Построение регулярного выражения по автомату: устранение вершин.
