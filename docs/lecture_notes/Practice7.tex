\section{Практика 7}

\subsection{Григорьев С.В.}

Алгоритмы на основе линейной алгебры.

Для реализации предлагается использовать следующие библиотеки. Так как с булевыми не везде хорошо, то будем использовать те типы, которые поддерживаются: \verb|Int|, \verb|float| и т.д.
\begin{itemize}
    \item Для языка Python --- \href{https://docs.scipy.org/doc/scipy/reference/sparse.html}{разреженные матрицы в scipy} и соответствующие опреации работы с ними: \href{https://docs.scipy.org/doc/scipy-0.14.0/reference/generated/scipy.sparse.kron.html}{\textbf{scipy.sparse.kron}} и \href{https://docs.scipy.org/doc/scipy/reference/generated/scipy.sparse.csr_matrix.html}{обычное матричное произведение}. Предпочтительный формат разреженных матриц --- CSR.
    \item Для языка Kotlin ---\href{http://la4j.org/}{la4j}. Операции: \href{http://la4j.org/apidocs/org/la4j/operation/ooplace/OoPlaceKroneckerProduct.html}{кронекер} и  \href{http://la4j.org/apidocs/org/la4j/operation/ooplace/OoPlaceMatricesMultiplication.html}{обычное умножение}.
\end{itemize}

Поэлементное сложение есть и там и там.

Для алгоритма на матричном умножении всё точно так же, как и в предыдущей ДЗ для Хеллингса.

Для тензорного произведения расширим формат представления входной грамматики.
Одна строка на нетерминал. Терминалы, нетерминалы, $\varepsilon$ обозначаются как и раньше.
Как и раньше, левая часть от правой отделена пробелом. В правой части можно использовать конструкции регулярных выражений: альтернатива, звезда клини, групперцющие скобки. Этот набор можно расширять по своему усмотрению.


Пример входа, описывающего граммтику $S \to (a S b)* \mid \varepsilon$:

\begin{verbatim}
S (a S b)* | eps
\end{verbatim}


Домашнее задание. Время на выполнение --- две недели. Один из алгоритмов --- на первую, оставшийся и эксперименты --- на вторую.
\begin{enumerate}
    \item Реализовать алгоритм, основанный на матричном умножении. На вход принимаются два файла: с граммтикой и входным графом. В результирующий файл печатается граммтика в ослабленной НФХ (с которой непосредсвенно работал алгоритм) и множество пар достижимых вершин для стартового нетерминала (одна пара на строку, две вершины через пробел).
    \item Реализовать алгоритм, основанный на тензорном произведении. На вход принимается файл с граммтикой и файл с графом. В результирующий файл печатается матрица смежности рекурсивного автомата (с которым непосредсвенно работал алгоритм, построчно, элементы разделены пробелом, пустая ячейка обозначается символом '.') и множество пар достижимых вершин для стартового нетерминала (одна пара на строку, две вершины через пробел).
    \item Сравнить производительность трёх реализованных алгоритмов (Хеллингс, матричное произведение, тензорное произведение). Результат --- описание эксперимента и таблица сравнения в readme.
\end{enumerate}
