\section{План занятий}
\begin{enumerate}
  \item Введение. О чём курс: общая структура, что будет и чего не будет. Правила получения оценки за курс. Базовые определения.
  \item Иерархия Хомского. Основные классы языков. За пределами иерархии Хомского. Нестроковые языки.
  \item Задача поиска пути с ограничениями в терминах формальных языков. Варианты постановки, прикладное значение, теоретические вопросы.
  \item Регулярные языки, конечные автоматы (детерминированные, недетерминированные), регулярные выражения. Операции над ними. Операции над автоматами как операции над их матрицами смежности. Поиск путей с регулярными ограничениями.
  \item Контекстно-свободные языки. Нормальная и ослабленная нормальная формы Хомского. Поиск путей с КС ограничениями. CYK и Hellings.
  \item Матричный алгоритм КС запросов.
  \item Тензорный алгоритм КС запросов.
  \item Дерево разбора и поиск путей. SPPF.
  \item Синтаксический анализ языков программирования. Лексика и синтаксис. Тонкости, проблемы, инструменты.
  \item ANTLR, LL, ещё раз про неоднозначности.
  \item Семантика языков программирования. Интерпретаторы. Что делать с деревом разбора.
  \item Атрибутные грамматики.
  \item Немного о том, что за КС тоже есть жизнь.
\end{enumerate}


Общая цель курса --- посмотреть на формальные языки с прикладной точки зрения. При этом предлагается попробовать применить их сразу в двух областях: классический синтаксический анализ языков программирования и анализ графов.

В ходе курса будет предложено разработать небольшой инструментарий для выполнения запросов к графам. Окажется, что алгоритмы для некоторых задач анализа графов непосредственно основаны на алгоритмах из теории формальных языков и синтаксического анализа. Далее, будет предложено разработать язык запросов, позволяющий использовать разработанные алгоритмы. Необходимо будет разработать сам язык, лексический и синтаксический  анализаторы для него, интерпретатор. Интерпретатор будет использовать разработанные алгоритмы выполнения запросов к графам.

Примерные темы задач с баллами.
\begin{enumerate}
  \item [5] Развернуть репозиторий, снабдить его всем необходимым (сборка, тесты). Научиться подгружать графы из набора данных, запрашивать у них вершины и рёбра. На основе этой функциональности реализовать первые тесты. Реализовать консольный интерфейс, позволяющий получить количество вершин и рёбер для указанного графа. Далее этот интерфейс будет расширяться и остальные задачи должны уметь работать с консолью.
  \item [2] Реализовать преобразование регулярного выражения в ДКА.
  \item [2] Реализовать преобразование графа в НКА.
  \item [5] Реализовать алгоритм выполнения регулярных запросов через тензорное произведение.
  \item [11] Сравнение производительности пересечения автоматов через тензорное произведение и стандартное из Pyformlang.
  \item [2] Реализовать преобразование контекстно-свободной грамматики в НФХ и ОНФХ.
  \item [5] Реализовать построение рекурсивного конечного автомата и его минимизацию.
  \item [5] Реализовать алгоритм синтаксического анализа CYK.
  \item [5] Реализовать алгоритм Хеллингса для задачи достижимости с контекстно-свободными ограничениями.
  \item [5] Реализовать матричный алгоритм (для булевых матриц) для задачи достижимости с контекстно-свободными ограничениями.
  \item [5] Реализовать тензорный алгоритм (для булевых матриц) для задачи достижимости с контекстно-свободными ограничениями.
  \item [15] Сравнение производительности алгоритмов для задачи достижимости с контекстно-свободными ограничениями: Хеллингс, матричный, тензорный.
  \item [5] Разработать конкретный синтаксис языка запросов к графам. Описать его в виде документации в репозитории. Снабдить примерами
  \item [5] Реализовать его парсер языка запросов к графам, согласно разработанному синтаксису.
  \item [3] Реализовать печать дерева разбора в DOT на основе стандартных возможностей ANTLR.
  \item [20] Реализовать интерпретатор языка запросов к графам.
\end{enumerate}
