\section{Практика 9}

\subsection{Григорьев С.В.}

Абстрактный и конкретный синтаксис языка запросов к графам, семантика языка запросов к гарфам.

Абстрактный синтаксис скрывает детали синтаксического анализа и конкретного ``текстового'' представления синтаксиса, оставляя только важную для дальнейшего анализа иформацию о структуре программы.

Первая, минималистичная, версия языка запросов к графам.

\begin{verbatim}
type script: Seq of List<stmt>

type stmt:
  | Connect of String
  | Select of obj_expr * graph_expr

type graph_expr =
  | Intersect of graph_expr * graph_expr
  | Query of pattern
  | GraphName of String

type obj_expr:
  | Edges
  | Count of obj_expr

type pattern:
  | smb of String
  | star of pattern
  | plus of pattern
  | alt of pattern * pattern
  | seq of pattern * pattern
  | option of pattern

\end{verbatim}

\newcommand{\sem}[1]{\llbracket #1 \rrbracket}

Тперь поговорим о семантике данного языка.

Семантика --- отображение из программ в семантический домен.
Программы обычно представляются как абстрактные синтаксические деревья.
Семантический домен --- то, что описывает смысл программы в контексте решаемой задачи.

Семантика бывает разной.
Классические примеры: динамическая семантика (вычисление программ) и статическая семантика (вывод типов).

Доменом динамической семантики арифметических выражений является множество функций из функции, сопоставляющей переменным целочисленные значения, в целые числа.

Семантика часто обозначается скобками $\llbracket$ и $\rrbracket$.

Семаника арифметических выражений:

$$\sem{\cdot} : (X \rightarrow \mathbb{N}) \rightarrow \mathbb{N}$$


\begin{align*}
  \sem{n}(\sigma) &= n, n \in \mathbb{N} \\
  \sem{v}(\sigma) &= \sigma(v) \\
  \sem{e_1 \oplus e_2}(\sigma) &= \sem{e_1}(\sigma) \otimes \sem{e_2}(\sigma) \\
  & \\
  \sem{-e}(\sigma) &= \sem{0-e}(\sigma)
\end{align*}

Здесь $\oplus$ --- конструкторы бинарных операций, $\otimes$ --- соответствующие арифметические функции.

В нашем случае, исполнение скрипта на получившимся языке достаточно естественно описывается состоянием из текущей базы данных, текущего графа, и текущего результата вычислений. Значение этой тройки и будут семантическим доменом нашего языка.

При обработке непустой последовательности инструкций сначала вычисляем результат выполнения первой инструкции, затем на нем --- всех остальных.

$$
\infer[]{c \xrightarrow{Seq \ (hd::tl)} c''}{c \xrightarrow{hd} c', \ \ c' \xrightarrow{Seq \ tl} c''}
$$

При обработке команды \verb|Connect| сохраняем её аргумент в конфигурации как путь к текущей базе данных.
$$
\infer[]{\langle db, graph, res\rangle \xrightarrow{Connect \ db'} \langle db', graph, res\rangle}{}
$$

При обработке инструкции \verb|Select| сперва вычисляется граф, из которого будут выбираться объекты, затем вычисляются сами объекты.

$$
\infer[]{c \xrightarrow{Select \ obj \ graph} c''}{c \xrightarrow{graph} c', \ \ c' \xrightarrow{obj} c''}
$$

При обработке инструкции \verb|Graph| загружаем в окружение граф из файла с именем, соотвествующим аргументу этой инструкции, лежащем в базе (директории), соответствующей значению \verb|cur_db|.

$$
\infer[]{\langle db, graph, res \rangle \xrightarrow{\textit{GraphName} \ name} \langle db ,loadGraph(String.concat(db,name)), res\rangle}{}
$$

При обработке инструкции \verb|Intersect| сперва вычисляем оба подвыражения, затем пересекаем получившиеся графы как конечные автоматы. Считаем все вершины стартовыми и финальными одновременно.

$$
\infer[]{\langle db, graph, res\rangle \xrightarrow{\textit{Intersect} \ graph_1 \ graph_2} \langle db, \textit{intersectFA}(graph', graph''), res\rangle}{\langle db, graph,res\rangle \xrightarrow{graph_1} \langle db, graph', res \rangle, \ \ \langle db, graph,res \rangle \xrightarrow{graph_2} \langle db, graph'', res \rangle}
$$


При обработке инструкции \verb|Query| необходимо построить минимальный детерминированный автомат по регулярному выражению, задаваемому шаблоном.

$$
\infer[]{\langle db, graph, res \rangle \xrightarrow{\textit{Query} \ pattern} \langle db, \textit{buildMinDFA}(pattern), res \rangle}{}
$$

При обработке инструкции \verb|Edges| в текущий результат вычислений записывается множество троек, соотвествующих рёбрам текущего графа.

$$
\infer[]{\langle db, graph, res \rangle \xrightarrow{\textit{Edges}} \langle db, graph, \{(v_i,e,v_j) \mid (v_i,e,v_j) \in graph.Edges\}\rangle}{}
$$

При обработке инструкции \verb|Count| сперва вычисляется подвыражене, а затем в текущий результат вычислений записывается размер результата вычисления подвыражения, если результат был множеством.

$$
\infer[type\_of(res') \equiv Set ]{\langle db, graph, res \rangle \xrightarrow{\textit{Count} \ expr} \langle db, graph, Set.count(res') \rangle}{\langle db, graph, res \rangle \xrightarrow{expr} \langle db, graph, res' \rangle}
$$

--------------------------------------------------------------------------------




Расширенная версия.

\begin{verbatim}
type script = Seq of List<stmt>

type stmt =
  | Connect of String
  | NamedPattern of String * pattern
  | Select of obj_expr * graph_expr

type graph_expr =
  | Intersect of graph_expr * graph_expr
  | Query of pattern
  | GraphName of String
  | SetStartAndFinal of vertices * vertices * graph_expr

type vertices =
  | Set of Set<int>
  | Range of int * int
  | None

type obj_expr =
  | Edges
  | Filter of cond * obj_expr
  | Count of obj_expr

type cond =
  | Cond of String * String * String * bool_expr

type bool_expr =
  | LblIs   of String * String
  | IsStart of String
  | IsFinal of String
  | And of bool_expr * bool_expr
  | Or of bool_expr * bool_expr
  | Not of bool_expr

type pattern =
  | Term of String
  | Nonterm of String
  | Star of pattern
  | Plus of pattern
  | Alt of pattern * pattern
  | Seq of List<pattern>
  | Option of pattern

\end{verbatim}


Для описания семантики для добавленных конструкций потребуется изменить домен --- в него необходимо добавить grammar (текущую граммтику).


При обработке инструкции \verb|NamedPattern| добавляем в \textit{grammar} правило, где левая часть --- имя шаблона, а правая часть --- регулятрное выражение построенное по шаблону.

$$
\infer[]{\langle db, graph, grammar, res \rangle \xrightarrow{\textit{NamedPattern} \ name \ pattern} \langle db, graph, \textit{addRule(grammar, name, pattern)}, res \rangle}{}
$$

Так как теперь запрос может быть не только регулярным, то изменится обработка кинструкции \verb|Query|. Теперь вместо детерминированного конечного автомата строится рекурсивный конечный автомат с учётом накопленной грамматики.

$$
\infer[]{\langle db, graph, grammar, res \rangle \xrightarrow{\textit{Query} \ pattern} \langle db, \textit{buildRSM}(pattern, grammar), grammar, res \rangle}{}
$$

Как следствие, изменяется обработка инструкции \verb|Intersect|: нам необхоимо отслеживать, какого типа графы пересекаются, так как для успешного выполнения операции необходимо, чтобы хотя бы один из них был обыкновенным конечным автоматом.

$$
\infer[\textit {$g'$ is DFA or $g''$ is DFA}]{\langle db, g, grm, res\rangle \xrightarrow{\textit{Intersect} \ g_1 \ g_2} \langle db, \textit{intersect}(g', g''), grm, res\rangle}{\langle db, g, grm, res\rangle \xrightarrow{g_1} \langle db, g', grm, res \rangle, \ \ \langle db, g, grm, res \rangle \xrightarrow{g_2} \langle db, g'', grm, res \rangle}
$$

Добавим операцию, позволяющую явно указывать стартовые и финальные состояния \verb|SetStartAndFinal|.

{
\scriptsize
$$
\infer[res' \neq \varnothing, res'' \neq \varnothing]{\langle db, g, grm, res\rangle \xrightarrow{\textit{SetStartAndFinal} \ start \ final \ g_1} \langle db, \textit{(g' with Start = $res'$, Final = $res''$)}, grm, res\rangle}{\langle db, g, grm, res\rangle \xrightarrow{g_1} \langle db, g', grm, res \rangle, \ \ \langle db, g', grm, res \rangle \xrightarrow{start} \langle db, g', grm, res' \rangle, \ \ \langle db, g', grm, res \rangle \xrightarrow{final} \langle db, g', grm, res'' \rangle}
$$
}

{
\scriptsize
$$
\infer[res' = \varnothing, res'' \neq \varnothing]{\langle db, g, grm, res\rangle \xrightarrow{\textit{SetStartAndFinal} \ start \ final \ g_1} \langle db, \textit{(g' with Final = $res''$)}, grm, res\rangle}{\langle db, g, grm, res\rangle \xrightarrow{g_1} \langle db, g', grm, res \rangle, \ \ \langle db, g', grm, res \rangle \xrightarrow{start} \langle db, g', grm, res' \rangle, \ \ \langle db, g', grm, res \rangle \xrightarrow{final} \langle db, g', grm, res'' \rangle}
$$
}

{
\scriptsize
$$
\infer[res' \neq \varnothing, res'' = \varnothing]{\langle db, g, grm, res\rangle \xrightarrow{\textit{SetStartAndFinal} \ start \ final \ g_1} \langle db, \textit{(g' with Start = $res$)}, grm, res\rangle}{\langle db, g, grm, res\rangle \xrightarrow{g_1} \langle db, g', grm, res \rangle, \ \ \langle db, g', grm, res \rangle \xrightarrow{start} \langle db, g', grm, res' \rangle, \ \ \langle db, g', grm, res \rangle \xrightarrow{final} \langle db, g', grm, res'' \rangle}
$$
}


{
\scriptsize
$$
\infer[res' = \varnothing, res'' = \varnothing]{\langle db, g, grm, res\rangle \xrightarrow{\textit{SetStartAndFinal} \ start \ final \ g_1} \langle db, g', grm, res\rangle}{\langle db, g, grm, res\rangle \xrightarrow{g_1} \langle db, g', grm, res \rangle, \ \ \langle db, g', grm, res \rangle \xrightarrow{start} \langle db, g', grm, res' \rangle, \ \ \langle db, g', grm, res \rangle \xrightarrow{final} \langle db, g', grm, res'' \rangle}
$$
}

$$
\infer[]{\langle db, graph, grammar, res \rangle \xrightarrow{\textit{Set} \ s} \langle db, g, grammar, s \cap \textit{graph.Vertices} \rangle}{}
$$


$$
\infer[]{\langle db, graph, grammar, res \rangle \xrightarrow{\textit{None}} \langle db, g, grammar, \varnothing \rangle}{}
$$

$$
\infer[]{\langle db, graph, grammar, res \rangle \xrightarrow{\textit{Range} \ x \ y} \langle db, g, grammar, \textit{graph.Vertices} \cap \{ \min(x,y) \ldots \max(x,y)\} \rangle}{}
$$

Расширим возможности обработки результата пересечения. Для этого добавим фильтрацию рёбер. Обработка конструкции \verb|Filter| выглядит следующим образом.

$$
\infer[\textit{type of res' is Set<Edge>}]{\langle db, graph, grammar, res \rangle \xrightarrow{\textit{Filter} \ cond \ src} \langle db, g, grammar, \textit{Set.filter cond res'} \rangle}{\langle db, graph, grammar, res \rangle \xrightarrow{src} \langle db, g, grammar, res' \rangle}
$$
