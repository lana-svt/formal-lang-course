\section{Практика 2}

\subsection{Григорьев С.В.}

Построение минимального ДКА по регулярному варажению.

Домашнее задание.
\begin{enumerate}
	\item Реализовать функцию (можно с применением библиотек), которая принимает на вход регулярное выражение в виде строки и строит по нему минимальный ДКА.
	\item Реализовать необходимые тесты на построение ДКА по регулярному выражению.
	\item Реализовать (можно с применением библиотек) пересечение минимального ДКА и НКА без $\varepsilon$-переходов.
	\item Реализовать необходимые тесты на пересечение ДКа и НКА.
\end{enumerate}



%Парсер регулярных выражений. Немного про AST и прочее.

%Пересечение ДКА и НКА (без $\varepsilon$-переходов): библиотека и руками через произведение Кронекера (прямое произведение). Сравнение производительности.

%Представление результата в виде графа и регулярного выражения.
