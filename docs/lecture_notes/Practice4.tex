\section{Практика 4}

\subsection{Григорьев С.В.}

Больше подробностей про рекурсивные автоматы: тотальная минимизация. Как их приментять для КС запросов. Тензоры + транзитивное замыкание.

Домашнее задание.
\begin{enumerate}
	\item Реализовать выполнение регулярных щапросов через тензорное произведение. Для тензорного произведения использовать существующие библиотеки линейной алгебры. Обратите внимание на то, что матрицы должны быть разреженными. Скорее всего, удобно будет использовать представление в виде набора булевых матриц.
	\item Интегрировать новую реализацию в клиент наравне со старой.
	\item Провести замеры производительности на графах из репозитория \url{https://github.com/JetBrains-Research/CFPQ_Data}. Графы брать из подпапки \verb|data/graphs/RDF|. Так как в графах присутствуют одинаковые отношения, то можно один и тот же запрос выполнять на всех графах. Отчёт оформить в виде раздела в  README репозитория в виде таблицы. Сравнить с результатми предыдущей задачи.

	\textit{Эти эксперименты проводятся локально! Не надо таскать репозиторий с графами за собой. Для тестов клиента использовать маленькие синтетические графы и запросы.}
	\item Реализовать необходимые тесты на работоспособность алгоритма через тензорное произведение.
\end{enumerate}
