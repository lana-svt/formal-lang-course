\section{Практика 11}

\subsection{Григорьев С.В.}


Дополнительные задачи на зачёт.

Для получения зачёта необходимо выполнить следующие условия.

\begin{enumerate}
\item Сдасть все семестровые задачи.
\item Если какие-то из задач в семестре были сданы с нарушением условий, установленных в начаое семестра, то необходимо решить дополнительные задачи. Какие именно, указано в \href{https://docs.google.com/spreadsheets/d/1g1ZVACS0ATW7pVpvSI6L4mLKsxyuJEVh_HX6xhQRdmI/edit#gid=0}{таблице с результатами}, в колонке ``Доп. задача''.
\end{enumerate}

Дополнительные задачи. В таблице указаны их номера. Красным выделены те, которые нужно решить.
Для каждой задачи должно быть представлено:
\begin{itemize}
	\item Расширение абстрактного синтаксиса (добавлено в документацию в репозитории)
	\item Расширенеи конкретного синтаксиса (расширена документация в репозитории)
	\item Примеры (добавлено в документацию в репозитории)
	\item Реализация
	\item Тесты
\end{itemize}
\begin{enumerate}
  \item Расширить команду \verb|list| так, чтобы можно было опционально указывать путь к БД, графы в которой хочется вывести. По умолчанию всё так же выводятся графы из текущей БД.
  \item Расширить команду \verb|list| так, чтобы с её помощью можно было вывести множество различных меток рёбер в указанном графе.

  \item Расширить команду \verb|select| возможностью опционально указывать алгоритм, с помощью которого выполнять текущий запрос. Воспроизвести эксперимент из ДЗ 5. Повлияло ли использование языка запросов на результаты экспериментов?

  \item Расширить команду \verb|select| возможностью в качестве графа-источника использовать результат операций объединения, пересечения и допллнения графов из текущей базы данных. То есть во \verb|from| можно птсать выражение над графами типа \verb|from| $\overline{(g_1 \cup g_2)} \cap g_3$. Данные операции должны трпктоваться как операции над языками, задаваемыми автоматом, где переходы определяются графом, а стартовые и финальные состояния --- все вершины графа.


\end{enumerate}
