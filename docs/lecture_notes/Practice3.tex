\section{Практика 3}

\subsection{Григорьев С.В.}

Пересечение автоматов --- это тензорное произведение матриц смежности. Пример.

Про коммутативность пересечения и некоммутативность тензорного произведения.

Домашнее задание.
\begin{enumerate}
	\item Реализовать консольный клиент, позволяющий
	\begin{enumerate}
		\item загрузить RDF-файл
		\item вывести список меток рёбер
		\item задать к загруженному графу регулярный запрос с возможностью указать представление результата: пустота ответа, автомат сдампить в файл в формате DOT (\url{https://www.graphviz.org/doc/info/lang.html}), пара (кол-во рёбер, кол-во вершин) в результирующем автомате
	    \item выйти из клиента.
    \end{enumerate}
	\item Подгрузку RDF и выполнение запросов реализовать на основе уже существующей функциональности.
	\item Провести замеры производительности на графах из репозитория \url{https://github.com/JetBrains-Research/CFPQ_Data}. Графы брать из подпапки \verb|data/graphs/RDF|. Так как в графах присутствуют одинаковые отношения, то можно один и тот же запрос выполнять на всех графах. Отчёт оформить в виде раздела в  README репозитория в виде таблицы.

	\textit{Эти эксперименты проводятся локально! Не надо таскать репозиторий с графами за собой. Для тестов клиента использовать маленькие синтетические RDF.}
	\item Реализовать необходимые тесты на работоспособность клиента.
\end{enumerate}
